\documentclass{beamer}
\usepackage[spanish]{babel}
\usepackage[utf8]{inputenc}
\usepackage{graphicx}

\newtheorem{definicion}{Definición}
\newtheorem{ejemplo}{Ejemplo}

%%%%%%%%%%%%%%%%%%%%%%%%%%%%%%%%%%%%%%%%%%%%%%%%%%%%%%%%%%%%%%%%%
\title[Presentación con Beamer]{Presentación con Beamer.}
\author[Tiffany López Nicholson]{Tiffany López Nicholson.}
\date[15 de marzo de 2013]{15 de marzo de 2013}
%%%%%%%%%%%%%%%%%%%%%%%%%%%%%%%%%%%%%%%%%%%%%%%%%%%%%%%%%%%%%%%%%
\usetheme{Antibes}
\usecolortheme[RGB={122,59,122}]{structure}
%%%%%%%%%%%%%%%%%%%%%%%%%%%%%%%%%%%%%%%%%%%%%%%%%%%%%%%%%%%%%%%%%

\begin{document}
%++++++++++++++++++++++++++++++++++++++++++++++++++++++++++++++++++++++++++++++  
\begin{frame}
  \titlepage

  \begin{scriptsize}
    \begin{center}
     Facultad de Matemáticas \\
     Universidad de La Laguna
    \end{center}
  \end{scriptsize}

\end{frame}
%++++++++++++++++++++++++++++++++++++++++++++++++++++++++++++++++++++++++++++++  

%++++++++++++++++++++++++++++++++++++++++++++++++++++++++++++++++++++++++++++++  
\begin{frame}
  \frametitle{Índice}  
  \tableofcontents[pausesections]
\end{frame}
%++++++++++++++++++++++++++++++++++++++++++++++++++++++++++++++++++++++++++++++  

\section{Utilizando Beamer.}

%++++++++++++++++++++++++++++++++++++++++++++++++++++++++++++++++++++++++++++++  
\begin{frame}

\frametitle{Utilizando Beamer.}

\begin{definicion}
  Para crear secciones es muy sencillo, utilizando comandos como en \LaTeX.
\end{definicion}

\begin{ejemplo}
  En esta presentación mostraré ejemplos de como se escriben fórmulas.
\end{ejemplo}

\end{frame}
%++++++++++++++++++++++++++++++++++++++++++++++++++++++++++++++++++++++++++++++  

\section{Fórmulas.}

%++++++++++++++++++++++++++++++++++++++++++++++++++++++++++++++++++++++++++++++  
\begin{frame}

\frametitle{Fórmulas.}

\begin{block}{Ejemplo}
  \begin{itemize}
  \item
  Fórmula de límites. 
  \pause

  \item
  Fórmula de la derivada con límites.
  \pause

  \item
  Fórmula para hallar la solución de una ecuación cuadrática.
  \pause

  \item
  Fórmula para hallar el área de un círculo.
  \pause

  \item
  Fórmula para hallar la longitud de una circunferencia.

  \end{itemize}
\end{block}

\end{frame}
%++++++++++++++++++++++++++++++++++++++++++++++++++++++++++++++++++++++++++++++  

  \subsection{Fórmula de límites y de la derivada.}

%++++++++++++++++++++++++++++++++++++++++++++++++++++++++++++++++++++++++++++++  
\begin{frame}
\frametitle{Fórmula de límites y de la derivada.}
\begin{itemize}
  \item
  Fórmula de los límites.\newline
  $lim_{x-a} f(x) = l$, siendo a un número real o infinito.\newline
  
  \item
  Fórmula de la derivada.\newline
  $f'(x)=lim_{h-0} \frac{f(x+h)-f(x)}{h}$.
\end{itemize}
\end{frame}
%++++++++++++++++++++++++++++++++++++++++++++++++++++++++++++++++++++++++++++++  

  \subsection{Fórmula para hallar la solución de una ecuación cuadrática.}

%++++++++++++++++++++++++++++++++++++++++++++++++++++++++++++++++++++++++++++++
\begin{frame}
\frametitle{Fórmula para hallar la solución de una ecuación cuadrática.}
  Sea $y= ax^2 + bx + c$, donde a no sea 0.
  Para hallar sus soluciones usamos:\newline
  \begin{center}
  $x_1 = \frac{-b+\sqrt{b^{2}-4ac}}{2a}$, $x_2 = \frac{-b-\sqrt{b^{2}-4ac}}{2a}$
  \end{center}
\end{frame}
%++++++++++++++++++++++++++++++++++++++++++++++++++++++++++++++++++++++++++++++

  \subsection{Fórmula para hallar el área de un círculo y la longitud de la circunferencia.}

%++++++++++++++++++++++++++++++++++++++++++++++++++++++++++++++++++++++++++++++
\begin{frame}
\frametitle{Fórmula para hallar el área de un círculo y la longitud de la circunferencia.}
  \begin{itemize}
  \item
  Fórmula para hallar el área.\newline
  Sea r el radio del círculo.
  \begin{center}
$A_c = \pi x r^2$
  \end{center}
  
  \item
  Fórmula para hallar la longitud.\newline
  \begin{center}
  $L_c = 2 x \pi x r$. 
  \end{center}
\end{itemize}
\end{frame}
%++++++++++++++++++++++++++++++++++++++++++++++++++++++++++++++++++++++++++++++

\section{Bibliografía}

%++++++++++++++++++++++++++++++++++++++++++++++++++++++++++++++++++++++++++++++
\begin{frame}
  \frametitle{Bibliografía}

  \begin{thebibliography}{10}

    \beamertemplatebookbibitems
    \bibitem[Manual LaTeX]{Man}  
    Manual de \LaTeX. {\small $http://www.fceia.unr.edu.ar/lcc/cdrom/Instalaciones/LaTex/latex.html$}

    \beamertemplatebookbibitems
    \bibitem[Cleaning up figure placement in Beamer]{Cleaning}  
    Cleaning up figure placement in Beamer. {\small $http://tex.stackexchange.com/questions/82662/cleaning-up-figure-placement-in-beamer-minipage$}

    \beamertemplatebookbibitems
    \bibitem[Blog de \LaTeX fino.]{latex} 
    Blog de \LaTeX fino. {\small $http://texnorte.blogspot.com.es/p/manual-de-latex.html$}

  \end{thebibliography}
\end{frame}


%++++++++++++++++++++++++++++++++++++++++++++++++++++++++++++++++++++++++++++++  
\end{document}